\documentclass[letterpaper,10pt]{article}

\usepackage{latexsym}
\usepackage[empty]{fullpage}
\usepackage{titlesec}
\usepackage{marvosym}
\usepackage[usenames,dvipsnames]{color}
\usepackage{verbatim}
\usepackage{enumitem}
\usepackage[hidelinks]{hyperref}
\usepackage{fancyhdr}
\usepackage[english]{babel}
\usepackage{multicol}
\usepackage{setspace}

\pagestyle{fancy}
\fancyhf{} % clear all header and footer fields
\fancyfoot{}
\renewcommand{\headrulewidth}{0pt}
\renewcommand{\footrulewidth}{0pt}

\newcommand{\highlightline}{\vspace*{-0.4em}}

% Adjust margins
\addtolength{\oddsidemargin}{-0.5in}
\addtolength{\evensidemargin}{-0.5in}
\addtolength{\textwidth}{1in}
\addtolength{\topmargin}{-.5in}
\addtolength{\textheight}{1.0in}


\usepackage{url, hyperref}
\hypersetup{
  colorlinks=true,
  linkcolor=cyan,
  filecolor=magenta,      
  urlcolor=blue,
}
\urlstyle{same}

\raggedbottom
\raggedright
\setlength{\tabcolsep}{0in}

% Sections formatting
\titleformat{\section}{
    \vspace{-4pt}\scshape\raggedright\large
    }{}{0em}{}[\color{black}\titlerule \vspace{-5pt}]
    
    %-------------------------
    % Custom commands
    \newcommand{\resumeItem}[2]{
        \item\small{
            \textbf{#1}{: #2 \vspace{-2pt}}
            }
            }
            
            \newcommand{\resumeSubheading}[4]{
                \vspace{-1pt}\item
                \begin{tabular*}{0.97\textwidth}[t]{l@{\extracolsep{\fill}}r}
                    \textbf{#1} & #2 \\
                    \textit{\small#3} & \textit{\small #4} \\
    \end{tabular*}\vspace{-5pt}
}

\newcommand{\resumeSubItem}[2]{\resumeItem{#1}{#2}\vspace{-4pt}}

\renewcommand{\labelitemii}{$\circ$}

\newcommand{\resumeSubHeadingListStart}{\begin{itemize}[leftmargin=*]}
\newcommand{\resumeSubHeadingListEnd}{\end{itemize}}
\newcommand{\resumeItemListStart}{\begin{itemize}}
\newcommand{\resumeItemListEnd}{\end{itemize}\vspace{-5pt}}

%-------------------------------------------
%%%%%%  CV STARTS HERE  %%%%%%%%%%%%%%%%%%%%%%%%%%%%

\begin{document}

%----------HEADING-----------------
\begin{tabular*}{\textwidth}{l@{\extracolsep{\fill}}r}
  \textbf{\Large Alexander Jansing} & Email : \href{mailto:alexander.jansing@gmail.com}{alexander.jansing@gmail.com}\\
  \href{https://github.com/apjansing}{https://github.com/apjansing} & Mobile : +1-315-601-8991 \\
\end{tabular*}

% --------HIGHLIGHTS------------
\section{Career Objective}
Data Scientist and Software Engineer looking for an organization where I can apply my 5 years of combined experience to make a positive impact in the world.

% % --------HIGHLIGHTS------------
% \section{Highlights}
% \resumeSubHeadingListStart
% \small
% \begin{multicols}{2}
%   \item 5+ years in data science and software engineering industry
%   % \highlightline\item Active DoD TS/SCI
%   \highlightline\item M.S. in Computer Science with a focus in Software Engineering and Mathematics
%   \highlightline\item B.S. in Applied Mathematics
%   \highlightline\item Operating Systems most comfortable with Linux, MacOS 
%   \columnbreak
%   \item Proficient with Java, Python, Bash
%   \highlightline\item Comfortable with Matlab, R, \LaTeX, Dart
%   \highlightline\item Familiar with HTML, CSS, Javascript, Scala, Groovy, OCaml
%   \highlightline\item Technologies comfortable with Git, Nifi, Hadoop, Spark, AWS, Maven, Docker, Docker-compose, ArgoCD, Kubernetes, MongoDB, Accumulo, Postgres, Jira, Confluence
% \end{multicols}
% \normalsize
% \resumeSubHeadingListEnd

%-----------EDUCATION-----------------
\section{Education}
\resumeSubHeadingListStart
% \resumeSubheading
% {Utica College}{Utica, NY}
% {Master of Science in Data Science}{Aug. 2020 -- Present}
\resumeSubheading
{SUNY Polytechnic Institute}{Utica, NY}
{Master of Science in Computer Science with a dual focus in Mathematics and Software Engineering}{Aug. 2015 -- May 2019}
\resumeItemListStart
\resumeItem{Final Project}{\href{https://github.com/apjansing/Open-House-Route-Planner}{Open House Route Planner}: A programmatic approach to finding a route to as many open houses on a given day. Project was inspired by a personal need when searching for a house.}
\resumeItem{GPA}{3.64}
% \resumeItem{Relevant Classwork}
% {Quantum Computing, AI Topic: Data Science, Machine Learning, Formal Methods, Big Data Platforms, Numerical Diff Equations}
\resumeItemListEnd
\resumeSubheading
{SUNY Oswego}{Oswego, NY}
{Bachelor of Science in Applied Mathematics}{Aug. 2012 -- May 2015}
\resumeSubHeadingListEnd

%-----------EXPERIENCE-----------------
\section{Experience}
\resumeSubHeadingListStart

\resumeSubheading
  {Assured Information Security}{Rome, NY}
  {Software Engineer III}{Feb 2021 - Present}
  \resumeItemListStart
    \resumeItem{Stargate}{Created Docker images and Kubernetes deployment manifests for a subset of the project and updated project to build in Linux environments.}
  \resumeItemListEnd

  \resumeSubheading
  {Booz Allen Hamilton}{Rome, NY}
  {Software Engineer, Implementation Specialist}{Oct 2018 - Feb 2021}
  \resumeItemListStart
    \resumeItem{VI2E - Pipeline Delivery}
    {Using Concourse, Docker, ArgoCD, Kubernetes, Python, and Bash scripts to create CI/CD pipelines for the Air Force's VI2E program.}
    \resumeItem{Swift}
    {Used Concourse, Sonarqube, Docker, Python, and Bash scripts to create CI/CD pipelines for the United States Air Force Research Laboratory.}
  \resumeItemListEnd  

\resumeSubheading
  {Lockheed Martin}{Liverpool, NY}
  {Software Engineer, Asc}{Mar 2018 - Sept 2018}
  \resumeItemListStart
    \resumeItem{SEWIP and Q-53 BEMA}
    {Developed analytics for noise reduction and identification of Modulation techniques using technologies like Matlab and Tensorflow.}
  \resumeItemListEnd

\resumeSubheading
  {Booz Allen Hamilton}{Rome, NY}
  {Data Scientist, Junior -- Computer Science}{Jan 2016 - Mar 2018}
  \resumeItemListStart
    \resumeItem{Active Insights}
    {Designed a data lake based using an Accumulo backend and OrientdDB for provenance tracking. ETL processes were performed with Apache Nifi.}
  \resumeItemListEnd

  \resumeSubheading
  {SUNY Polytechnic}{Utica, NY}
  {Graduate Assistant}{Aug 2015 - Jan 2016}
  \resumeItemListStart
    \resumeItem{Finite Mathematics}
    {Graded homework, held office hours, and designed grading schemes for Finite Mathematics.}
  \resumeItemListEnd

\resumeSubheading
  {New York State Air National Guard}{Syracuse, NY}
  {MQ-9 Avionics Journeyman}{Apr 2012 - Apr 2018}
  \resumeItemListStart
    \resumeItem{$\mathbf{174^{th}}$ AMXS}{Performed scheduled maintenance and troubleshot errors on avionics systems on MQ-9 Reaper.}
  \resumeItemListEnd

\resumeSubheading
  {United States Air Force}{Luke AFB, AZ}
  {F-16 Avionics Journeyman}{Oct 2009 - Apr 2012}
  \resumeItemListStart
    \resumeItem{$\mathbf{756^{th}}$ AMXS - 310 AMU}{Performed scheduled maintenance and troubleshot errors on avionics systems on F-16 fighter jets. Systems responsible for included flight controls, IFF, Radar warning receiver, FCR, JHMCS, and others.}
  \resumeItemListEnd  
  
  \resumeSubHeadingListEnd
  
% --------PROGRAMMING SKILLS------------
  \section{Programming Skills}
  \resumeSubHeadingListStart
    \item{\textbf{Languages}}{: Java, Python, Matlab, R, SQL, Scala, Groovy, OCaml, Bash, \LaTeX, HTML, CSS, Javascript}
    
    \item{\textbf{Technologies}}{: Git, Nifi, Hadoop, Spark, Concourse, Maven, Docker, Docker-compose, MongoDB, Accumulo, Postgres, OrientDB, TitanDB, AWS, Jira, Confluence}
    
    \item{\textbf{Operating Systems}}{: Linux, MacOS}
  \resumeSubHeadingListEnd

%-----------PROJECTS-----------------
\section{Projects}
\resumeSubHeadingListStart
\resumeSubItem{Graduate School Final Project --- \href{https://github.com/apjansing/Open-House-Route-Planner}{Open House Route Planner}}
{A project that allows the user to provide a series of calendar events and returns several routes one could take to visit as many open houses as possible. Project was written in Python, uses the \emph{Esri API} for geocoding and route finding, \emph{MongoDB} for caching of locations, and \emph{Docker-compose} for infrastructure. Idea originally worked on during hack Upstate (see Projects section).}

\resumeSubItem{Hack Upstate XI --- Open House Route Planner*}
{\footnotesize Javascript, Docker, MongoDB, Esri API  \hfill \emph{Grand Prize} and \emph{Esri API Prize} \\ \normalsize A minimum viable product for a hackathon that took lat-long locations and found the optimal route to visit all points using the Esri API. Project was primarily written in Javascript, uses the \emph{Esri API} for geocoding and route finding, and \emph{Docker-compose} for infrastructure. Time-boxing of open houses and travel times were touched upon, but not fully implemented.\\ \tiny * No code survived between hackathon and graduate school.}

\resumeSubItem{Hack Mohawk Valley --- Move Helper}
{\footnotesize HTML, CSS, Javascript, MongoDB  \hfill \emph{Best Use of Open Data} \\ \normalsize An app to help you get the info you need to move into a new area. Data returned includes sources from Syracuse Open Data crime, lead, and code violations.}

\resumeSubItem{Hack Upstate X --- Buffalo Crime Data}
{\footnotesize Python, MongoDB \hfill \emph{Best Use of Open Data} \\ \normalsize Hackathon project to discover crimes that occurred a specified distance away from police cameras and plotted them to show clusters of crimes in the city of Buffalo.}
\resumeSubHeadingListEnd

%-----------Publications-----------------
\section{Publications}
\resumeSubHeadingListStart

\resumeSubheading
  {Medium}{}
  {}{April 2020 - Present}
  \resumeItemListStart
    \resumeItem{Writing a Custom Concourse Resource --- (Four Parts)}
    {
    \begin{enumerate}
      \item --- Overview \\ An overview of all the components involved in writing your own Concourse resource. This and the following articles were inspired by the process it took to write my first resource as the tutorials available were not sufficient.
      \item --- the \textsf{check} \\ The tutorial goes over how the \textsf{check} component detects new versions at an endpoint and how to use that in the \textsf{in} component of the resource.
      \item --- the \textsf{in} \\ The tutorial goes over how the \textsf{in} uses the output of the \textsf{check} component to fetch data from an endpoint, save that data, what needs to be output, and how it interacts with the \textsf{out}.
      \item --- the \textsf{out} \\ The tutorial goes over how the \textsf{out} component delivers data to an endpoint and how it runs the \textsf{in} component to fetch the data the \textsf{out} just wrote.
    \end{enumerate}
    }
  \resumeItemListEnd

\resumeSubHeadingListEnd

 %-------------------------------------------
 \end{document}